\documentclass[ngerman]{gdb-aufgabenblatt}

\usepackage{enumitem}
\renewcommand{\Aufgabenblatt}{1}
\renewcommand{\Ausgabedatum}{Mi. 18.10.2017}
\renewcommand{\Abgabedatum}{Fr. 03.11.2017}
\renewcommand{\Gruppe}{Meimerstorf,Jochens,T�ter}
\renewcommand{\STiNEGruppe}{19}


\begin{document}

\section{Informationssysteme}
\subsection{Charakterisierung}
Ein Informationssystem ist ein informationstechnologisch unters�tztes Anwendungssystem, also im Endeffekt ein Computersystem, welches betriebliche Aufgaben ausf�hrt. Charaktisierend f�r ein Datenbanksystem ist eine leichte Handhabbarkeit der Daten, die Kontrolle der Datenintegrit�t und die Kontrolle �ber die operationalen Daten, au�erdem im heutigen Status in Zeiten von Amazon und google ist auch die Leistung und Skalierbarkeit der Daten wichtig. Die Skalierbarkeit, da eine Datenbank f�r eine kleine Webanwendung sowie auch f�r ein gro�es DB-System wie Amazons Nutzerverwaltung funktionieren muss. 
\textbf{Aufgaben:} 
\begin{enumerate}[label=\arabic*)]
\item Organisieren von Arbeitsabl�ufen \\
\item Paralleler Zugriff von mehreren Stellen \\
\item Hoher Durchsatz und kurze Antwortzeit \\
\end{enumerate}
\subsection{Datenunabh�ngigkeit}
Die Datenunabh�ngigkeit bezeichnet die Unabh�ngikeit der Daten und den dazugeh�rigen Anwendungsprogrammen. Realisiert wird dies meist, dadurch das ein DMS, eine Schnittstelle zu den Daten bereitstellt welche die Anwendungsprogramme dann verwenden k�nnen. \\ Daraus folgt auch direkt die physische Unabh�ngikeit, welche darstellt, das die Anwendungsprogramme nicht ver�ndert werden m�ssen wenn sich die physische Speicherungsart der Daten auf der Festplatte �ndert. Da sich die Schnittstelle nicht �ndert. \\
Die logische Unabh�ngikeit bezeichnet, die Robustheit der Anwendungsprogramme gegen �nderungen im Datenbankschema. Dies ist auf jeden Fall bei SQL Datenbanken nicht der Fall, da wenn ich mein Schema ver�ndere ich auch meine SQL-Queries ver�ndern muss. \\ 
\subsection{Beispiele}
\begin{enumerate}
\item SAP System in Unternehmen \\
\item Googles Suchmaschine \\	
\item 
\end{enumerate}
\section{Miniwelt}
\subsection{relevante Objekte}
\subsection{Anforderungen}
\section{Transaktionen}
\subsection{Zeitpunkt A}
\subsubsection{Dateisystem}
\begin{enumerate}
\item Daten im RAM: \\
\item Daten auf Platte: \\
\end{enumerate}
\subsubsection{Datenbanksystem}
\begin{enumerate}
\item Daten im RAM: \\
\item Daten auf Platte: \\
\end{enumerate}
\subsection{Zeitpunkt B}
\subsubsection{Dateisystem}
\begin{enumerate}
\item Daten im RAM: \\
\item Daten auf Platte: \\
\end{enumerate}
\subsubsection{Datenbanksystem}
\begin{enumerate}
\item Daten im RAM: \\
\item Daten auf Platte: \\
\end{enumerate}
\section{Warm-Up SQL}
\subsection{Create/Insert}
Mit dem Befehl CREATE TABLE wird eine Tabelle mit dem Namen user in der Datenbank gdb$\_$Gruppe erstellt. Diese enh�lt dem Befehl entsprechend drei verschiedene Eintr�ge, eine ID welche als prim�rer Key f�r die Tabelle verwendet wird, au�erdem einen 49 Zeichen langen Character der nicht leer sein darf wenn ein Objekt in die Tabelle eingef�gt wird, und als letztes noch das passwort, welches 8 Zeichen lang sein kann und auch nicht leer sein darf. \\
Mit dem zweiten Befehl INSERT INTO wird ein Datensatz bestehend aus (ID,Name,Passwort) in die vorher erstelle Tabelle eingef�gt.\\
\subsection{SELECT/DROP}
Mit dem Befehl SELECT * FROM table WHERE, werden alle Datens�tze der Tabelle table ausgegeben bei denen die Bedingung welche nach WHERE steht, stimmt. Also in unserem Fall genau unseren Datensatz. Hierbei ist zu beachten, dass durch das * wir alle Spalten des Datensatzes mitgeliefert bekommen und nicht nur eine spezielle Spalte \\
Mit dem Befehl DROP k�nnen Tabellen komplett aus der Datenbank gel�scht werden, wobei hier darauf zu achten ist das DROP tats�chlich im Gegensatz zu TRUNCATE die Tabelle l�scht und nicht nur den Inhalt der Tabelle. \\




\end{document}