\documentclass[ngerman]{gdb-aufgabenblatt}
\usepackage{enumitem}
\usepackage{dingbat}
\usepackage{ifsym}
\usepackage{amssymb}
\usepackage{amsmath}
\renewcommand{\Aufgabenblatt}{4}
\renewcommand{\Ausgabedatum}{Mi. 29.11.2017}
\renewcommand{\Abgabedatum}{Fr. 15.12.2017}
\renewcommand{\Gruppe}{Meimerstorf,Jochens,T�ter}
\renewcommand{\STiNEGruppe}{19}
\usepackage{tabularx}
\usepackage[normalem]{ulem}

\begin{document}

\section{Relationenalgebra}
\subsection{}
Die Nachnamen aller Matrosen, die auf einem Schiff anheuern, dessen Heimathafen Hamburg ist. \\
\subsection{}
$\pi_{Name,Stapellauf}((Matrose \verbund{MNR=Matrose} (\sigma_{Jahresgehalt>400000}anheuern) \verbund{Schiff=SNR} Schiff))$
\subsection{}
$\pi_{Nachname}\sigma_{Heimathafen=Ausbildungsort}((Matrose \verbund{MNR=Matrose} anheuern \verbund{Schiff=SNR} Schiff))$
\subsection{}
$\pi_{Grundsteinlegung}(Hafen\verbund{HNR=HNR}(\pi_{HNR}(Hafen) \setminus (\pi_{Ausbildungsort}(Matrose))))$
\section{Schemadefinition}
\begin{verbatim}
CREATE TABLE Person (
    Nachname VARCHAR(50) NOT NULL,
    Vorname VARCHAR(50) NOT NULL,
    Geburtsdatum DATE ,
    Wohnort VARCHAR(50) NOT NULL,
    Lieblingsbuch INT,
    BID INT, 
    CONSTRAINT prim_person  PRIMARY KEY(Nachname,Vorname),
    CONSTRAINT age_check CHECK(Geburtsdatum <  GETDATE())
);

CREATE TABLE Bibliothek (
    BID INT PRIMARY KEY,
    Name VARCHAR(50),
    Adresse VARCHAR(50),
	CONSTRAINT uq_name_adress UNIQUE(Name, Adresse),
	Leiter_Nachname VARCHAR(50) NOT NULL,
	Leiter_Vorname VARCHAR(50) NOT NULL,
	CONSTRAINT FK_PersonBibliothek FOREIGN KEY (Leiter_Nachname,Leiter_Vorname) REFERENCES Person(Nachname,Vorname)
);
ALTER TABLE Person
ADD CONSTRAINT fk_bib_person FOREIGN KEY (BID) REFERENCES Bibliothek(BID);

CREATE TABLE Buch (
    ISBN INT PRIMARY KEY,
    Titel VARCHAR(50),
    Autor_Nachname  VARCHAR(50) NOT NULL,
    Autor_Vorname  VARCHAR(50) NOT NULL,
    CONSTRAINT FK_Autor FOREIGN KEY (Autor_Nachname,Autor_Vorname) REFERENCES Person(Nachname,Vorname)
);

ALTER TABLE Person
ADD CONSTRAINT fk_lieblingsbuch FOREIGN KEY (Lieblingsbuch) REFERENCES Buch(ISBN);

CREATE TABLE leiht_aus (
	Vorname VARCHAR(50),
	Nachname VARCHAR(50),
	ISBN INT,
    CONSTRAINT FK_leightaus_Buch FOREIGN KEY (ISBN) REFERENCES Buch(ISBN),
    CONSTRAINT FK_leihtaus_person FOREIGN KEY (Nachname,Vorname) REFERENCES Person(Nachname,Vorname)
);
CREATE TABLE in_Bestand (
	BID INT NOT NULL,
	ISBN INT,
	Anzahl INT,
    CONSTRAINT FK_inBestand_Buch FOREIGN KEY (ISBN) REFERENCES Buch(ISBN),
    CONSTRAINT FK_inBestand_Bib FOREIGN KEY (BID) REFERENCES Bibliothek(BID),
    CONSTRAINT Anzahl_Check Check(Anzahl>1)
);
\end{verbatim}

\section{SQL}
\subsection{}
\begin{verbatim}
SELECT DISTINCT m1.Geburtsdatum
FROM Matrose m1, Schiff s1, anheuern a1
Where m1.MNR = a1.Matrose 
AND a1.Schiff = s1.SNR 
AND a1.Dienstbeginn = ?1957 ? 04 ? 01?
ORDER m1.Geburtsdatum DESC
\end{verbatim}
\subsection{}
\begin{verbatim}
SELECT *
FROM Matrose
Where Geburtsdatum > ?1530 ? 01 ? 01? 
AND Geburtsdatum < ?1535 ? 01 ? 01? 
AND Nachname LIKE 'H%'
\end{verbatim}
\subsection{}
\begin{verbatim}
SELECT m1.MNR,m1.Nachname,MAX(a1.Jahresgehalt)
FROM Matrose m1, anheuern a1
Where m1.MNR = a1.Matrose
GROUP BY a1.Matrose
\end{verbatim}
\subsection{}
\begin{verbatim}
SELECT h.Ort
FROM Hafen h
WHERE h.HNR NOT IN (
SELECT s.Heimathafen
FROM Schiff)
\end{verbatim}
\subsection{}
\begin{verbatim}
SELECT m.MNR, MAX(a1.Jahresgehalt)
FROM Matrose m, anheuern a1, Schiff s1
WHERE m.MNR = a1.Matrose AND a1.Schiff = s1.SNR AND a1.Dienstbeginn > '2016 ? 12 ? 24'
\end{verbatim}
\section{}
Der Ort des Hafens, an dem zwei Schiffe mit unterschiedlichen Heimath�fen am selben Tag Stapellauf hatten.
 
\end{document}